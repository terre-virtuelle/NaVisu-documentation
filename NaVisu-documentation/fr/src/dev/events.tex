\chapter{Gestion des �v�nements}


\section{Architecture}
\subsection{Principes}
1- Pour les objets en overlay, c'est du JavFX 

1.1 charg� depuis un fichier FXML :

quit.setOnMouseClicked((MouseEvent event) -> {
	instrument.off();
});

quit est ici un Button.

1.2 H�ritant de Widget2DController, par exemple :

public class RouteEditorController
extends Widget2DController

Il y a d�j� un comportement par d�faut, apparition, disparition, d�placement, ... Voir le d�tail dans la classe Widget2DController, losr de l'instanciation on peut personnaliser le widget : 

routeEditorController = new RouteEditorController(this,
layersManagerServices ,
KeyCode.M, KeyCombination.CONTROL\_DOWN);
La touche M, permettra de faire disparaitre ce widget.

2- Pour les objets venant de WorldWind, c'est du Swing, il suffit de mettre l'objet WorldWindow � l'�coute d'un �v�nement de s�lection par exemple et de filtrer cet evt : 

2.1 Globalement :
\begin{verbatim}
private void addListeners() {
	wwd.addSelectListener((SelectEvent event) -> {
		Object o = event.getTopObject();
		if (event.isLeftClick() && o != null) {
			if (o.getClass() == PointPlacemark.class) {
				System.out.println("OK");
			}
		}
	});
}
\end{verbatim}
On voit ici que tous les objets de type PointPlacemark, au moins pour les couches sensibles aux �v�nements, seront s�lectionn�s.
Pour r�agir � cet evt, cel� peut convenir pour afficher leur nom par exemple, mais pour pour une action sp�cifique, c'est sans doute trop g�n�ral. Il faut donc filtrer des objets dont la vue est un PointPlacemark, mais qui ont un type particulier.

Une solution : cr�er la classe Poimark : 
\begin{verbatim}
public class Poimark extends PointPlacemark{
	
	public Poimark(Position pstn) {
		super(pstn);
	}
}
\end{verbatim}
Ensuite filtrer les Poimark : 
\begin{verbatim}
private void addListeners() {
	wwd.addSelectListener((SelectEvent event) -> {
		Object o = event.getTopObject();
		if (event.isLeftClick() && o != null) {
			if (o.getClass() == Poimark.class) {
				System.out.println("Action");
			}
		}
	});
}
\end{verbatim}
Enfin instancier des Poimark plut�t que de simples PointPlacemark :

\begin{verbatim}
public Set<Pair<Double, Double>> showPoi(Map<Pair<Double, Double>, String> data) {
	Set<Pair<Double, Double>> latLonSet = data.keySet();
	
	latLonSet.stream().map((ll) -> {
		String imageAddress = NavigationIcons.ICONS.get(ICON_NAME);
		
		Poimark poimark = new Poimark(Position.fromDegrees(ll.getX(), ll.getY(), 0));
		poimark.setAltitudeMode(WorldWind.CLAMP_TO_GROUND);
		poimark.setValue(AVKey.DISPLAY_NAME, data.get(ll));
		PointPlacemarkAttributes normalAttributes = new PointPlacemarkAttributes();
		normalAttributes.setImageAddress(imageAddress);
		normalAttributes.setScale(0.4);
		poimark.setAttributes(normalAttributes);
		return poimark;
	}).forEach((placemark) -> {
	sailingDirectionsIconsLayer.addRenderable(placemark);
});

return latLonSet;
}
\end{verbatim}